% 第5章
\chapter{绘制图表}
图表的制作大概是\LaTeX 中最令人着迷的部分了,为图表编写的宏包、工具、书籍、文档数不胜数,\LaTeX 在这方面所能达到的效果也从简单的直线稿图、简单表格发展到极为复杂的图形图像、数据报表,其功能不亚于许多专业图表软件。但另一方面,缺乏直观的代码也让不少人视为畏图。让图表问题变得更容易,是许多\LaTeX 用户的愿望。这一章我们要进入这个全新的领域,从基本的工具开始,渐次发散开来,逐步领略个中妙趣。
\section{\LaTeX 中的表格}
在语义上,表格的作用是展示多种相关的内容,在形式上,表格是按行和列对齐的一组内容。表格是二维延伸的特殊排版对象,与tabbing环境简单地预设对齐位置不同,在表格的较后面内容的宽度也会影响前面内容的排列。在\LaTeX 中,表格是逐行输入的,可以设置表格的列对齐格式和表格线,通过一些特殊的宏包还可以达成一些特殊的效果。
\subsection{tabular 和array }
\subsection{表格单元的合并与分割}
\verb|\multicolumn{<项数>}{<新列格式>}{<内容>}|
命令可用于将一行中几个不同的项合并为一项,它经常用于排版跨列的表头,例如:
\begin{table}[ht]
\begin{minipage}{0.7\linewidth}
\begin{lstlisting}
\begin{tabular}{|r|r|}
\hline
  \multicolumn{2}{|c|}{成绩}\\ \hline
  语文& 数学\\ \hline
  80 & 100\\ \hline
  \hline
\end{tabular}
\end{lstlisting}
\end{minipage}
	\quad
	\begin{tabular}{|r|r|}
		\hline
		\multicolumn{2}{|c|}{成绩}\\ \hline
		语文& 数学\\ \hline
		80 & 100\\ \hline
	\end{tabular}
\end{table}

注意这里合并的新列格式里面只能有一个\verb|c、l、r或p{<宽>}|,以及可选的\@选项和表线。\verb|\multicolumn|会重定义它所产生的列后面的竖线(如果是第一列,也包括前面的竖线),当表格有竖线时,\verb|\multicolumn|命令增加或减少应用的竖线,当然,也可以用它来产生间断的竖线。\verb|\multicolumn|命令不仅可以用于合并多列,也可以用来只“合并”一列,作用是改变所在表项的对齐、竖线格式,例如:
\begin{table}[ht]
	\begin{minipage}{0.7\linewidth}
		\begin{lstlisting}
\begin{tabular}{|r|r|}
\hline
\multicolumn{1}{|c|}{输入}&
\multicolumn{1}{|c|}{输出} \\ \hline
1& 1\\5 &25\\15 & 225\\ \hline
\end{tabular}
		\end{lstlisting}
	\end{minipage}
	\quad
	\begin{tabular}{|r|r|}
		\hline
		\multicolumn{1}{|c|}{输入}&
		\multicolumn{1}{|c|}{输出} \\ \hline
		1& 1\\5 &25\\
		15 & 225\\ \hline
	\end{tabular}
\end{table}
\verb|\cline|命令与\verb|hline|命令类似,都用来画水平的表格线。不过\verb|\cline|带有一个形如
\section{插图和变换}
\subsection{浮动体和标题控制}
\subsubsection{浮动体}
\subsubsection{标题控制和caption宏包}
\subsubsection{并排与子图表}
在实际中,经常需要把好几个图表并列放在一起输出。由于\LaTeX 的加以限制,所以只要直接把图表放在一个浮动体里面就可以了,例如:
\begin{table}[ht]
\begin{minipage}{0.7\linewidth}
\begin{lstlisting}
\begin{tabular}{|r|r|}
 \hline
 \multicolumn{1}{|c|}{输入}&
 \multicolumn{1}{|c|}{输出} \\ \hline
 1& 1\\5 &25\\15 & 225\\ \hline
 \end{tabular}
\end{lstlisting}
\end{minipage}
	\quad
	\begin{tabular}{|r|r|}
		\hline
		\multicolumn{1}{|c|}{输入}&
		\multicolumn{1}{|c|}{输出} \\ \hline
		1& 1\\5 &25\\
		15 & 225\\ \hline
	\end{tabular}

\end{table}

\begin{lstlisting}[numbers=left]  
\begin{table}
\centering
\caption{`并排的表格`}
\begin{tabular}{|c|c|}
\hline `图` & `表` \\ \hline 
\end{tabular} 
\qquad
\begin{tabular}{|c|c|}
\hline Figure & Table \\ \hline A & B \\ \hline
\end{tabular}
\end{table}
\end{lstlisting}               % 插入代码块

\begin{table}[ht]
	\centering
	\caption{并排的表格}
	\begin{tabular}{|c|c|}
		\hline 图 & 表 \\ \hline 
	\end{tabular}
	\qquad
	\begin{tabular}{|c|c|}
		\hline Figure & Table \\ \hline A & B \\ \hline
	\end{tabular}
\end{table}

\subsubsection{浮动控制与float宏包}




\subsection{使用彩色}
基本的彩色工具是color宏包,它是\LaTeX 的基本组件,graphics工具包的一部分。

在color宏包中,使用彩色的基本命令是 {\color{blue}{\verb|\colr|}} 和 {\color{blue}{\verb|\textcolor|}} :
\begin{quote} %引用环境
	\begin{verbatim}
	\color{<颜色>}
	\textcolor{<颜色>}{<文字>}
	\end{verbatim}  % 抄写环境
\end{quote} 
\noindent 它们的语法格式和字体选择命令相似,\verb|\color|是声明命令(同一分组内)后面的内容都使用指定的颜色输出,而 \verb|\textcolor|则将参数〈文字〉以指定颜色输出,例如:
\begin{table}[h]
	\begin{minipage}{0.5\linewidth}
		\begin{lstlisting}
% \usepackage{color}
\color{red}`红色文字夹杂`%
\textcolor{blue}{`蓝色`}`文字`
		\end{lstlisting}
	\end{minipage}
	\quad
	\begin{tabular}{|l|}
		\hline
		\color{red}红色文字夹杂%
		\textcolor{blue}{蓝色}文字\\
		\hline
	\end{tabular}
\end{table}