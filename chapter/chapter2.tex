%Chapter2
%第二章,不能单独编译
\chapter{组织你的文本}

\section{文字和符号}
\label{sec:wenzi}
字体的设置是一个麻烦的东西,英文字体的字体坐标是字体族、字体形状、字体系列,这样三维的方式。
中文没有这些。而且还有各种各样的编码问题,以Windows出现的问题最多。


\begin{itemize}
%\item {\hyqh 汉仪旗黑 \par 
\begin{quote}
abcdefghijlmnopqrstuvwxyz

ABCDEFGHIJKLMNOPQRSTUVWXYZ

这是一段中英字体表示样例
\end{quote}}
%\item {\CJKfamily{syst} 思源宋体 \par 
\begin{quote}
abcdefghijlmnopqrstuvwxyz

ABCDEFGHIJKLMNOPQRSTUVWXYZ

这是一段中英字体表示样例
\end{quote}}
\item {\CJKfamily{FZShuTi} 方正舒体 \par 
\begin{quote}
abcdefghijlmnopqrstuvwxyz

ABCDEFGHIJKLMNOPQRSTUVWXYZ

这是一段中英字体表示样例
\end{quote}}
\item {\CJKfamily{hwhp} 华文琥珀 \par 
\begin{quote}
abcdefghijlmnopqrstuvwxyz

ABCDEFGHIJKLMNOPQRSTUVWXYZ

这是一段中英字体表示样例
\end{quote}}
\item { 正文 \par 
\begin{quote}
abcdefghijlmnopqrstuvwxyz

ABCDEFGHIJKLMNOPQRSTUVWXYZ

这是一段中英字体表示样例
\end{quote}}
\item {\bfseries 粗体 \par 
\begin{quote}
abcdefghijlmnopqrstuvwxyz

ABCDEFGHIJKLMNOPQRSTUVWXYZ

这是一段中英字体表示样例
\end{quote}}
\item {\itshape 斜体 \par 
\begin{quote}
abcdefghijlmnopqrstuvwxyz

ABCDEFGHIJKLMNOPQRSTUVWXYZ

这是一段中英字体表示样例
\end{quote}}
\item {\bfseries\itshape 粗斜体 \par 
\begin{quote}
abcdefghijlmnopqrstuvwxyz

ABCDEFGHIJKLMNOPQRSTUVWXYZ

这是一段中英字体表示样例
\end{quote}}
\end{itemize}

在windows系统中,由于汉字编的问题,CMD命令行输出的中文会乱码,在终端键入chcp,显示编码类别是936,通过设置语言打开{设置}{时间和语言}{语言}{管理语言设置},勾选``BEta版:使用Unicode UTF-8'',重启之后,终端键入chcp,显示编码类别是65001。

更改之后发现无法好好编译了。tex系统无法找到方正字体,交叉引用也无法好好编译了

也许设置这个也可以解决中文路径无法编译的问题,蒽,猜测错误
\section{段落和文本环境}
 \subsection{正文段落}
 
 换行可以用\verb|\par|,段落之间隔行可以用\verb|\bigspace|
\subsection{文本环境}
在Markdown中{\color{red}列表}和{\color{red}{引用}}是常用的标记方式,在\LaTeX 中这些可以使用文本环境来实现

引用环境有quote环境和quotation环境,它们的区别在于是否首行缩进。

verse环境来排版诗歌韵文,我还么用过。

摘要环境abstract是写论文常用的环境,可以设置一个小标题,这个小标题可以重新定义\verb"\abstractname"来设置
\subsection{列表环境}

列表环境有编号的enumerate环境、不编号的itemize环境和带关键字的description环境。都是使用\verb|\item|开始一个列表项。
\section{文档的结构层次}

\subsection{多文件编译}

使用\verb|\include|命令来插入章节文件。

\section{}