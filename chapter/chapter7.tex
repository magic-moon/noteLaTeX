% 第7章
\chapter{从错误中救赎}

\section[换行]{How to add an empty line between paragraphs?}

如何在段落之间添加空行?不像Markdown 那样容易,如果使用 \verb`\\` 强制换行,也能实现想要的效果,但是\LaTeX 会给出警告,这对于强迫症患者而言是无法容忍的。

使用 \verb`\\`强制换行,\LaTeX 给出了以下警告:
\begin{lstlisting}
Underfull \hbox (badness 10000)
\end{lstlisting}

最终我选择使用 \verb|\bigskip|  命令实现了换行,完美的解决了这个问题,在此感谢强大的Google 搜索引擎,以及无私奉献的国外网友。
\section{字体}

仿宋字体在我的Ubuntu系统中不存在,这使得编译给出了警告
\begin{lstlisting}
Font "FandolFang-Regular" does not contain requested
(fontspec)	Script "CJK".
\end{lstlisting}
需解决

\section{windows系统下使用LaTeX问题与解决}

在windows系统下使用LaTeX文件名和路径都不能使用中文,因此又要诸多配置,所以啊,学习LaTeX还是使用安卓手机最为方便。不过这里还是记录一下在windows系统下使用LaTeX的情况。

能解决的把问题和解决方法都记录一下,不能解决的先把问题放下稍后解决。

在Windows系统下,我的工具是

LaTeX发行版:TeXLive 2019
编辑器:VS Code

\subsection{遇到的问题}
    

\begin{enumerate}
\item \LaTeX 文件中文名和中文路径无法编译

虽然可以在中文路径编译了,却没法使用正向和反向查找,是否中文路径和是否中文文件名,最后较好的方案是非中文路径下中文文件名,能够实现编译和正反查找。

\item fc-list命令显示出来的汉字乱码\label{ite:luanma},这个问题我找到了答案,具体回答可参考\ref{sec:wenzi}节关于字体的介绍

\item 编译的时候出现\verb|\ref|未定义

解决,原来是我的编译方法除了问题,在没有bib文件的情况下,使用bibtex编译出现的错误,因为引用是需要aux文件的,因此此情况下只用使用xelatex编译两遍就行了。但是我设置编译方式为自动两次xelatex编译又会出错,不知道是何原因。只能手动来编译两次方无问题。再次实验后这个问题也得到了解决。

\item Redefining CJKfamily `\CJKrmdefault' (SimSun(0)).

没有找到解决办法,这里不影响输出结果,放弃来
\item 使用synctex进行反向定位,参数有 `-synctex=1/-1/0’(ctrl+逗号打开配置文件,右键修改相关参数):
\item 
\end{enumerate}

\subsection{一些有趣的地方}

vs code编辑器有个直接的终端模块,我可以在里面使用命令行,不用再打开CMD了,乱码问题\ref{ite:luanma}依旧存在。




