\chapter{用手机编写\LaTeX 文档}

由于疫情的原因,我在床上的时间不少。使用电脑就不方便,我就开发了用手机编写\LaTeX 文档的技能。其实还挺方便的,让我看感觉比电脑还好用。需要三个工具,Termux来安装环境,QuickEdit来编写文本,Adobe Acrobat作为阅读工具。

\section{宏包管理}

termux提供了texlive的最简化安装,对于初学者而言,我反对完整安装,其耗费时间且难以理解,不利于初学者的学习。我建议搭建以最快的速度安装,马上尝试\LaTeX 带来的优势。

apt install texlive

只需要这一步就完成了最简化的安装,只需要两分钟。

当完成第一步安装之后开始

\section{有一些基本包需要安装}
hello!世界。

第一步:安装了lm 包,这是拉丁字体包,有了它可以使用英文了。

第二步:安装xecjk包

第三步:安装ctex包。自动安装了ms、ttfutils、ulem、zhnumber。

第四步:安装l3kernel、l3packages。自动安装了l3backend。

第五步:安装fontspec。自动安装了euenc、fontspec、iftex、tipa、xunicode。

第六步:安装fandol。

以上包可以使用一条命令安装,但我建议你一个个来,遇到问题再去安装新的,

完成了以上安装之后,编写汉语文档基本没有问题了。

我想增加『大纲』这一项,有的地方叫目录,实际上使用的是标签。遇到以下错误:

\section{手机用户要注意的地方}

我强烈建议大家买一个8英寸以上的平板来使用,分辨率要高,这是阅读PDF电子书的利器。比电脑不知道方便了多少倍。

下面我来讲使用手机和平板写\LaTeX 要注意的地方。

第一条,最重要,排版。这是最重要的,强烈建议不要使用A4页面排版,A5页面排版对于平板是恰当的选择,使用手机来阅读也勉强可以,但仍然有点费眼。我再次建议大学生要买平板,可以装卡的那种,替换掉你那对学习没有帮助的手机吧。


\section{使用手机编译的好处}

无时无刻,随想随写。

打造个人知识管理体系。

手机阅读,方便回溯。

\subsection{出现问题与解决}
\begin{lstlisting}[frame=leftline, xleftmargin=19pt]
kpathsea: Running mktexmf pzdr ! I can't find file 'pzdr'. 

<*> \mode:=ljfour; mag:=1; nonstopmode; input pzdr
\end{lstlisting}

安装了zapfding解决了问题。texlive manager在安装确实包时很好用,但是一旦更新,就会出现各种错误,旧包,最好是先卸载掉,然后再安装上新包。实在没办法,可以把texlive完全卸载再重新安装。

然后又有以下问题

\begin{lstlisting}[
         language=tex,
         basicstyle=\footnotesize\ttfamily,% 基本风格
         numbers=left,    % 行号
         numbersep=10pt,  % 行号间隔 
         tabsize=2,       % 缩进
         extendedchars=true, %扩展符号?
         breaklines=true, % 自动换行
         frame=leftline,  % 框架左边竖线
         xleftmargin=19pt,% 竖线左边间距
         showspaces=false,% 空格字符加下划线
         showstringspaces=false,% 字符串中的空格加下划线
         showtabs=false,  % 字符串中的tab加下划线
 ]
! Package ifluatex Error: Name clash, \ifluatex is already defined.

See the ifluatex package documentation for explanation.

Type  H <return>  for immediate help.
\end{lstlisting}

错误

\begin{lstlisting}[
         language=tex,
         basicstyle=\footnotesize\ttfamily,% 基本风格
         numbers=left,    % 行号
         numbersep=10pt,  % 行号间隔 
         tabsize=2,       % 缩进
         extendedchars=true, %扩展符号?
         breaklines=true, % 自动换行
         frame=leftline,  % 框架左边竖线
         xleftmargin=19pt,% 竖线左边间距
         showspaces=false,% 空格字符加下划线
         showstringspaces=false,% 字符串中的空格加下划线
         showtabs=false,  % 字符串中的tab加下划线
 ]
! I can't find file `simplekv.tex'.
<to be read again>
                   \relax
l.77 \fi

(Press Enter to retry, or Control-D to exit)
\end{lstlisting}


\section{修改字体}

\begin{lstlisting}
pkg install fontconfig-utils
fc-list
$ pwd
/data/data/com.termux/files/usr/share/fonts
$ cp /storage/FAD8-30D6/mywiki/assets/Fonts/SourceHanSerifSC-Regular.otf ./ %把字体移动到当前路径
$ ls
SourceHanSerifSC-Regular.otf  TTF
$ fc-list :lang-zh
/data/data/com.termux/files/usr/share/fonts/SourceHanSerifSC-Regular.otf: Source Han Serif SC,思源宋体:style=Regular
/data/data/com.termux/files/usr/share/fonts/TTF/DejaVuSerifCondensed.ttf: DejaVu Serif,DejaVu Serif Condensed:style=Condensed,Book
\end{lstlisting}
查看当前系统字体。使用fc-list :lang-zh
显示出来字体

发现在termux里可以配置字体路径,通过把字体路径改到sdcard路径下,方便管理字体,把windows系统上的大部分字体转移到平板上。我通过在/sdcard下新建了一个fonts文件夹来放置所需的字体。

配置文件所在的路径是

\verb|/data/data/com.termux/files/usr/etc/fonts|

font-config.json具体配置文件如下:
\begin{lstlisting}[language=html]
<dir>/system/fonts</dir>
<dir>/data/data/com.termux/files/usr/share/fonts</dir>
<dir>/storage/FAD8-30D6/Fonts</dir>
\end{lstlisting}

\section{在\LaTeX 中使用字体}

使用xecjk宏包的\verb'\setCJKfamilyfont{hwhp}{华文琥珀}'命令,

或者fontspec 宏包的 \verb|\newfontfamily<命令>[(可选项)]{<字体名>}|

使用思源宋体:%{\syst 这是一段思源宋体文字} 

普通Times New Roman

%{\hyqh 这是一段汉仪旗黑文字}

%{\CJKfamily{syst} 思源宋体}

{\CJKfamily{FZShuTi} 方正舒体}

{\CJKfamily{hwhp} 华文琥珀字体}
