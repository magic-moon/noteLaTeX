% uft8 %
\usepackage{amsmath,amssymb, amsfonts}

\usepackage{chemfig}
\usepackage{graphicx}
\usepackage{makeidx}         %调用makedidx宏包
\usepackage{listings}        %用于插入代码块
\usepackage{
			xcolor,        %为代码块设置颜色
			fontspec,
			}

\lstset{
	columns=fixed, 
    breaklines=true, % 自动换行 
    numbers=left,  % 在左侧显示行号
    basicstyle=\footnotesize\ttfamily,% 基本风格
	frame=single,                                       % 不显示背景边框
	tabsize=2,       % 设置缩进的单位为2
	backgroundcolor=\color[RGB]{245,245,244},           % 设定背景颜色
	keywordstyle=\color[RGB]{40,40,255},                % 设定关键字颜色
	numberstyle=\footnotesize\color{darkgray},          % 设定行号格式
	commentstyle=\it\color[RGB]{0,96,96},               % 设置代码注释的格式
	stringstyle=\rmfamily\slshape\color[RGB]{128,0,0}, 	% 设置字符串格式
	showstringspaces=false,                             % 不显示字符串中的空格
	language=teX,                                       % 设置语言
	%escapeinside=``									% 逃逸字符,浮动体无法直接使用汉字,使用一个逃逸字符夹住汉字,即可使用
}							%

\setcounter{secnumdepth}{4}								%设置标题的等级深度为4


%生成标签,就是大纲也是目录	
\usepackage[bookmarkstype=toc,
	bookmarksopen=false, %设置标签是否打开
	breaklinks,
	colorlinks,
	linkcolor=black,
	citecolor=red,
	urlcolor=blue,
	bookmarksnumbered=true,
	bookmarksopenlevel=4,
	pdftitle={note LaTeX}, %设置文档标题
	pdfauthor={张中魁}, %设置文档作者
	]{hyperref}						

\ctexset{chapter ={
		beforeskip = 0pt, 
		fixskip = true, 
		format = \Huge\bfseries\raggedright, 
		nameformat = \rule{\linewidth}{2bp}\par\bigskip\hfill\chapternamebox,
		number = \arabic{chapter}, 
		aftername = \par\medskip, 
		aftertitle = \par\bigskip\nointerlineskip\rule{\linewidth}{2bp}\par}
		}												%章节格式设置

\newcommand\chapternamebox[1]{% 
	\parbox{\ccwd}{\linespread{1}\selectfont\centering #1}} 

\usepackage[a5paper,left=1cm,right=1cm,top=1.5cm,bottom=1.5cm]{geometry}
%%%%%字体设置区
\setmainfont{Times New Roman}
\setsansfont{Verdana}
\setmonofont{Courier New}

%\CJKrmdefault{宋体}

\setCJKmainfont[BoldFont=黑体,ItalicFont=楷体,BoldItalicFont=隶书]{SimSun}   
\setCJKfamilyfont{FZShuTi}{方正舒体}
\setCJKfamilyfont{hwhp}{华文琥珀}

%\setCJKfamilyfont{hyqh}{汉仪旗黑}
%\setCJKfamilyfont{syst}{思源宋体}

%\newcommand\hyqh{\CJKfamily{hyqh}}
%\newcommand{\syst}{\CJKfamily{syst}} 
